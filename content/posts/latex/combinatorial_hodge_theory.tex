\documentclass[a4paper]{article}
\usepackage[margin=.8in]{geometry}
\usepackage{graphicx}
\usepackage{multicol}
\usepackage{float}
\usepackage{tikz}
\usepackage{amsmath, amsthm, amsfonts, amssymb}

\author{Kevin Shu}
\title{Combinatorial Hodge Theory}

\usepackage[parfill]{parskip}

%Define theorem formatting
\newtheorem{theorem}{Theorem}
\newtheorem{lemma}{Lemma}
\newtheorem{def}{Definition}

\newcommand{\R}{\mathbb{R}}
\newcommand{\C}{\mathbb{C}}
\newcommand{\N}{\mathbb{N}}
\newcommand{\Z}{\mathbb{Z}}
\newcommand{\Q}{\mathbb{Q}}
\newcommand{\E}{\mathbb{E}}
\newcommand{\Hom}{\textbf{\text{Hom}}}
\newcommand{\PP}{\textbf{P}}
\newcommand{\SPACE}{\textbf{SPACE}}
\newcommand{\NP}{\textbf{NP}}
\newcommand{\SAT}{\textbf{SAT}}
\newcommand{\pard}[2]{\frac{\partial #1}{\partial #2}}
\DeclareMathOperator*{\argmin}{arg\,min}
\DeclareMathOperator*{\argmax}{arg\,max}
\newcommand{\st}{{\text{ s.t. }}}

\begin{document}
\maketitle
\begin{def}
    A \textbf{fan}, $\Sigma$ is a collection of convex cones in $\R^n$ so that if $\sigma \in \Sigma$, and $\tau$ is a face of $\sigma$, then $\tau \in \Sigma$. Also, if $\tau, \sigma \in \Sigma$, then $\tau \cap \sigma$ must be a common face of both $\tau$ and $\sigma$. In some ways, this is like a projective version of a simplicial complex.
\end{def}

Let $\Sigma$ be a complete simplicial fan in $V$. $\mathcal{A}(\Sigma)$ is then the ring of conewise polynomial functions on $\Sigma$. 

\textit{Aside} One way to think of a conewise polynomial function is that we assign a polynomial $p_{\sigma}$  on each cone $\sigma \in \Sigma$, so that if $\tau$ is a face of $\sigma$, then $p_{\sigma} |_{\tau} = p_{\tau}$, where $|_{\tau}$ denotes the restriction of $p$ to the subset $\tau$. Notice the Zariski closure of a cone is the affine subspace it spans. If we think of the fan whose maximal cones are the coordinate subspaces in $\R^2$, then an example of a conewise polynomial function would be given by 

\begin{figure}[htpb]
    \centering
    \includegraphics[width=0.8\linewidth]{name.ext}
    \caption{Name}%
    \label{fig:name}
\end{figure}

\end{document}
